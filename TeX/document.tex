\documentclass[letter,12pt]{article}
\usepackage{graphicx}
\usepackage{url}
\usepackage{setspace}
\usepackage{ragged2e}
\usepackage{hanging}
\usepackage{hyperref}
\usepackage{fancyhdr}
\usepackage{lipsum}
\usepackage[none]{hyphenat}
\usepackage{enumitem}
\usepackage{amsmath}

\makeatletter
\newcommand{\distas}[1]{\mathbin{\overset{#1}{\kern\z@\sim}}}
\newsavebox{\mybox}\newsavebox{\mysim}
\newcommand{\distras}[1]{%
	\savebox{\mybox}{\hbox{\kern3pt$\scriptstyle#1$\kern3pt}}%
	\savebox{\mysim}{\hbox{$\sim$}}%
	\mathbin{\overset{#1}{\kern\z@\resizebox{\wd\mybox}{\ht\mysim}{$\sim$}}}%
}
\makeatother

\fancyhead{}
\fancyhead[L]{DISC THROW VARIANCE}
\fancyhead[R]{\thepage}
\fancyfoot{}
\fancypagestyle{plain}{
	\fancyhead{}
	\fancyhead[L]{Running head: DISC THROW VARIANCE}
	\fancyfoot{}
	\fancyhead[R]{\thepage}
}
\pagestyle{fancy}
\renewcommand{\headrulewidth}{0pt}
\addtolength{\headwidth}{\marginparsep}
\addtolength{\headwidth}{\marginparwidth}

\doublespacing

\addtolength{\oddsidemargin}{-.5in}
\addtolength{\evensidemargin}{-.5in}
\addtolength{\textwidth}{1in}

\addtolength{\topmargin}{-.875in}
\addtolength{\textheight}{1.75in}

\begin{document}
	\thispagestyle{plain}
	\begin{center}
		\vspace*{150pt}
		Evaluating Variance in Distance of Disc Throws by Throw Type and Disc Type. \\
		Joshua Katz \& Sejin Kim \\
		Kenyon College \\
	\end{center}
	\vspace*{120pt}
	\centering{Author Note} \\
	\begin{raggedright}
		Joshua Katz \& Sejin Kim, Kenyon College. \\
		Contact: \href{mailto:katz1@kenyon.edu}{katz1@kenyon.edu}, \href{mailto:kim3@kenyon.edu}{kim3@kenyon.edu} \\
		\vspace*{40pt}
		\textit{Keywords:} analysis of variance, ANOVA, frisbee, disc throwing
	\end{raggedright}
	
	\newpage

	\begin{center}
		Evaluating Variance in Distance of Disc Throws by Throw Type and Disc Type.
	\end{center}
	\begin{center}
		\textbf{I. Introduction}
	\end{center}
	\justify
	%PUT THE TEXT HERE, THIS LIPSUM IS FOR SOME FILLER
	This project was inspired by a curiosity that Josh had: is there a statistically significant difference in the distance that a frisbee would fly, depending on how you threw it? It seems obvious: yes, certain types of throws would go further. But, how could this be quantified? The only “hand-wavey” evidence that either of us had seen was the real-world performance in ultimate frisbee games and practices. Moreover, there are different types of discs, too. Disc golf discs are shaped differently, and thus have different flight patterns. Would some significant variance arise from this? Because of this, our group was interested in exploring the variance between the three main types of throws, and between two types of discs.

	We wanted would like to evaluate the throw distance of different types of throwing discs using two factors: by the type of disc, a disc golf disc (a “disc”), and a standard USA Ultimate frisbee (a “Frisbee”), and by the throw type, backhand, flick, and hammer. The end dataset should include 30 observations, five of each variant.  Instead of only measuring the vertical displacement, we also thought that certain throws may have an appreciable amount of horizontal displacement. Therefore, Wwe measured each throw in two-dimensional2D Euclidean Distance, taking into account horizontal displacement as well as vertical distance.  Total Ddistance waswill be measured in meters, and do not need normalized, since all measures will be done from the same scale. We expect the data to be roughly normal, if not skewed slightly higher. All throws will be done in approximately the same environmental conditions to minimize unnecessary noise.

	It is worth noting that to properly eliminate unnecessary throw-to-throw variation, we should use some sort of throwing machine. However, to get such a machine that could throw all three types would be prohibitively expensive (if one even exists in such a configuration), and thus, Josh will play the role of throwing machine. In order to prevent certain throws from being ranked lower due to exhaustion, the next throw will be determined by a random number generator, using the R script \verb|sample(1:30)|, and assigning each throw a unique ID.

	\begin{center}
		\textbf{II. Materials and Procedures}
	\end{center}
	\justify
	%PUT THE TEXT HERE, THIS LIPSUM IS FOR SOME FILLER
	\lipsum[14-17]
	
	\begin{center}
		\textbf{III. Analysis}
	\end{center}
	\justify
	%PUT THE TEXT HERE, THIS LIPSUM IS FOR SOME FILLER
	\lipsum[18]
	
	\begin{center}
		\textbf{a. Exploring the data graphically}
	\end{center}
	\justify
	%PUT THE TEXT HERE, THIS LIPSUM IS FOR SOME FILLER
	\lipsum[19-22]
	
	\begin{center}
		\textbf{b. Analysis of variance}
	\end{center}
	\justify
	%PUT THE TEXT HERE, THIS LIPSUM IS FOR SOME FILLER
	\lipsum[23-26]
	
	\begin{center}
		\textbf{c. Conditions for the ANOVA}
	\end{center}
	\justify
	%PUT THE TEXT HERE, THIS LIPSUM IS FOR SOME FILLER
	\lipsum[27-29]
	
	\begin{center}
		\textbf{IV. Results and the Model}
	\end{center}
	\justify
	%PUT THE TEXT HERE, THIS LIPSUM IS FOR SOME FILLER
	\lipsum[30-33]
	
	\begin{center}
		\textbf{V. Further Discussion}
	\end{center}
	\justify
	%PUT THE TEXT HERE, THIS LIPSUM IS FOR SOME FILLER
	\lipsum[34-36]
	
	\newpage
	
	\begin{center}
		References
	\end{center}
	\raggedright
	\begin{hangparas}{.5in}{1}
		Citation needed.
	\end{hangparas}
\end{document}
